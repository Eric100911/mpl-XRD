%!xelatex = 'xelatex --halt-on-error %O %S'

\documentclass{thuemp}
\begin{document}

% 标题,作者
\emptitle{}
\empauthor{王驰}{王合英}

% 奇数页页眉 % 请在这里写出第一作者以及论文题目
\fancyhead[CO]{{\footnotesize 王驰:多晶X射线衍射的物相分析及其应用}}


%%%%%%%%%%%%%%%%%%%%%%%%%%%%%%%%%%%%%%%%%%%%%%%%%%%%%%%%%%%%%%%%
% 关键词 摘要 首页脚注
%%%%%%%%关键词
\Keyword{}
\twocolumn[
\begin{@twocolumnfalse}
\maketitle

%%%%%%%%摘要
\begin{empAbstract}
\end{empAbstract}

%%%%%%%%英文标题、作者、摘要、关键词
\emptitleEn{Experiments of Modern Physics in Tsinghua University}
\empauthorEn{Chi Wang}{Heying Wang}
\KeywordEn{keyword1, keyword2, keyword3, keyword4, keyword5}

\begin{empAbstractEn}
\end{empAbstractEn}

%%%%%%%%首页角注,依次为实验时间、报告时间、学号、email
\empfirstfoot{2024-09-29}{2025-06-09}{2022012259}{chi-wang22@mails.tsinghua.edu.cn}
\end{@twocolumnfalse}
]
%%%%%%%%!首页角注可能与正文重叠,请通过调整正文中第一页的\enlargethispage{-3.3cm}位置手动校准正文底部位置:
%%%%%%%%%%%%%%%%%%%%%%%%%%%%%%%%%%%%%%%%%%%%%%%%%%%%%%%%%%%%%%%%
%  正文由此开始
\wuhao 
%  分栏开始

\section{引言}
\enlargethispage{-3.3cm}

\section{实验内容}

本实验使用X射线衍射仪,对提供的各晶体粉末样品进行X射线衍射分析,测绘出X射线衍射谱,并使用Jade 9软件对所得衍射谱进行寻峰,结合【PDF卡片】进行分析比对,最终确定物质组成。所使用的X射线仪中【如图...】,X射线管以及探测器可相对于样品托架以大小相等、方向相反的角速度转动;在此过程中,X射线管发射出波长一定的X射线,经过托架上的样品衍射后,部分X射线进入探测器,其强度与衍射角$\theta$关系即被记录下来,形成X射线衍射谱。本实验所采用的X射线管靶材为$\text{Cu}$,使用特征谱线$\text{Cu~K\alpha}$($\lambda = 1.54059~\angstrom$)设置电压$38~\text{kV}$,电流$10~\text{mA}$,在X射线管与探测器夹角$2\theta$取$20^\circ \~ 120^\circ$的区间内,选择步长$0.03^\circ$,对样品X射线衍射谱进行测绘。

本实验所使用的样品中,已知化学组分的有多晶硅粉末($\text{Si}$)、氯化钠粉末($\text{NaCl}$)、氯化钠单晶($\text{Si}$)、铜($\text{Cu}$),钼($\text{Mo}$),石墨烯粉末($\text{C}$)、石墨粉末($\text{C}$)、金刚石粉末($\text{Si}$),另有一份白色粉末状试样化学成分尚待本次实验测定。

\section{实验结果与分析}

\subsection{多晶硅的X射线衍射谱测量}

在本部分实验对多晶硅粉末的晶格常数进行测定,并据此确证实验仪器以及方法的可靠性。

在晶体衍射中,对各个晶面以及相应晶面间距$d$,其衍射峰对应衍射角$\theta$满足Bragg公式:

\begin{equation}
    2d\sin\theta  = \lambda
\end{equation}

\section{结论}


%%%%%%%%%%%%%%%%%%%%%%%%%%%%%%%%%%%%%%%%%%%%%%%%%%%%%%%%%%%%%%%%
%  参考文献
%%%%%%%%%%%%%%%%%%%%%%%%%%%%%%%%%%%%%%%%%%%%%%%%%%%%%%%%%%%%%%%%
%  参考文献按GB/T 7714-2015《文后参考文献著录规则》的要求著录. 
%  参考文献在正文中的引用方法:\cite{bib文件条目的第一行}

\renewcommand\refname{\heiti\wuhao\centerline{参考文献}\global\def\refname{参考文献}}
\vskip 12pt


\let\OLDthebibliography\thebibliography
\renewcommand\thebibliography[1]{
  \OLDthebibliography{#1}
  \setlength{\parskip}{0pt}
  \setlength{\itemsep}{0pt plus 0.3ex}
}

{
\renewcommand{\baselinestretch}{0.9}
\liuhao
\bibliographystyle{gbt7714-numerical}
\bibliography{./Report/TempExample}
}

\appendix
\section{衍射谱寻峰数据}


\end{document}
