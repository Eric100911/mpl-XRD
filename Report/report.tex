%!xelatex = 'xelatex --halt-on-error %O %S'

\documentclass{thuemp}
\begin{document}

% 标题,作者
\emptitle{}
\empauthor{王驰}{王合英}

% 奇数页页眉 % 请在这里写出第一作者以及论文题目
\fancyhead[CO]{{\footnotesize 王驰:多晶X射线衍射的物相分析及其应用}}


%%%%%%%%%%%%%%%%%%%%%%%%%%%%%%%%%%%%%%%%%%%%%%%%%%%%%%%%%%%%%%%%
% 关键词 摘要 首页脚注
%%%%%%%%关键词
\Keyword{}
\twocolumn[
\begin{@twocolumnfalse}
\maketitle

%%%%%%%%摘要
\begin{empAbstract}
\end{empAbstract}

%%%%%%%%英文标题、作者、摘要、关键词
\emptitleEn{Experiments of Modern Physics in Tsinghua University}
\empauthorEn{Chi Wang}{Heying Wang}
\KeywordEn{keyword1, keyword2, keyword3, keyword4, keyword5}

\begin{empAbstractEn}
\end{empAbstractEn}

%%%%%%%%首页角注,依次为实验时间、报告时间、学号、email
\empfirstfoot{2024-09-20}{2025-06-09}{2022012259}{chi-wang22@mails.tsinghua.edu.cn}
\end{@twocolumnfalse}
]
%%%%%%%%!首页角注可能与正文重叠,请通过调整正文中第一页的\enlargethispage{-3.3cm}位置手动校准正文底部位置:
%%%%%%%%%%%%%%%%%%%%%%%%%%%%%%%%%%%%%%%%%%%%%%%%%%%%%%%%%%%%%%%%
%  正文由此开始
\wuhao 
%  分栏开始

\section{引言}
\enlargethispage{-3.3cm}

\section{实验内容}

本实验使用X射线衍射仪,对提供的各晶体粉末样品进行X射线衍射分析,测绘出X射线衍射谱,并使用Jade 9软件对所得衍射谱进行寻峰,结合【PDF卡片】进行分析比对,最终确定物质组成。

本实验所采用的X射线管靶材为$\text{Cu}$,使用特征谱线$\text{Cu~K\alpha}$

\section{实验结果与分析}


\section{结论}


%%%%%%%%%%%%%%%%%%%%%%%%%%%%%%%%%%%%%%%%%%%%%%%%%%%%%%%%%%%%%%%%
%  参考文献
%%%%%%%%%%%%%%%%%%%%%%%%%%%%%%%%%%%%%%%%%%%%%%%%%%%%%%%%%%%%%%%%
%  参考文献按GB/T 7714-2015《文后参考文献著录规则》的要求著录. 
%  参考文献在正文中的引用方法:\cite{bib文件条目的第一行}

\renewcommand\refname{\heiti\wuhao\centerline{参考文献}\global\def\refname{参考文献}}
\vskip 12pt


\let\OLDthebibliography\thebibliography
\renewcommand\thebibliography[1]{
  \OLDthebibliography{#1}
  \setlength{\parskip}{0pt}
  \setlength{\itemsep}{0pt plus 0.3ex}
}

{
\renewcommand{\baselinestretch}{0.9}
\liuhao
\bibliographystyle{gbt7714-numerical}
\bibliography{./Report/TempExample}
}

\appendix
\section{衍射谱寻峰数据}


\end{document}
